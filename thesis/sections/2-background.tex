\section{Background and Related Work}
\label{sec:background}

\cmtDK[inline]{7 pages}

\subsection{Language Games}
% \begin{itemize}
%     \item why language games?
%           \begin{itemize}
%               \item origins of language
%               \item flexibility
%               \item efficiency
%           \end{itemize}
%     \item setup of language game
%           \begin{itemize}
%               \item cooperative vs self-interested
%               \item asymetry
%               \item gumbel softmax vs reinforce
%               \item EGG
%               \item bottleneck
%           \end{itemize}
%     \item how does emerged language look?
%           \begin{itemize}
%               \item compositionality
%               \item generalization
%               \item efficiency
%               \item referring expressions
%               \item discriminative vs description
%               \item entropy minimization
%           \end{itemize}
%     \item why relating objects?
% \end{itemize}

The center of this thesis evolves around language games between deep neural models.
Hereby, multiple deep neural agents need to solve a task, by communicating with symbols, which initially are not associated with any meaning.
By using and interpreting these symbols in there communication, the agents start to give these symbols and their combinations meaning.
After this process, a new artificial language emerged, with which the agents can communicate with each other.

In section \ref{sec:aims_languages}, I will explain reasons, why research in this field is conducted and how it may help in further research.
Section \ref{sec:setup-lg} shows in more detail, how the language games are set up and how exactly agents can communicate and the language emerges.
In section \ref{sec:properties-el}, I will discuss, how the emerged languages can be built up.
Finally, section \ref{sec:referring} discusses, how the language games will be used in this thesis, to explore how agents can refer to objects in images.

\subsubsection{Aims of emerged languages}
\label{sec:aims_languages}
The setup of language games allows for very controlled rules of how agents can behave.


\subsubsection{Setup of language games}
\label{sec:setup-lg}
The term of 'language games' was first introduced by \citet{Wittgenstein1953}.
The author describes language games as uses of language between multiple interlocutors.
These can be any small parts of conversations, for instance between a teacher and a student teaching a new concept or between two persons, discussing a topic.
The rules in each of these situations differ and therefore the meanings and semantics of words and sentences may differ from situation to situation.
An interjection 'Water!' may be a warning, an answer to a question, a request or something else, depending on the context.
The meanings are bound to the rules of the language games.

This reasoning was taken up, when trying to train artificial entities to produce a language.
One of the original games is the signaling game, proposed by \citet{Lewis1969}.
In this game, a sender needs to send a message to a receiver, based on information only available to the sender.
Based on the message, the receiver proposes an action.
Both sender and receiver are rewarded in the same way if the proposed action was correct.
Hence, both agents need to invent a language together, fit to the conditions of the game.

The entities can be set into real or simulated interactive situation, a language game, that meaning and a language can emerge (\citet{Kirby2002}).
Multiple deep neural models, called agents, communicate with a set of symbols to solve a task.
Guided by the task and the rules, how agents can generate and interpret symbols, a language can emerge.

% This reasoning was taken up, when trying to train deep neural models to produce language.
% These models need to be set into an interactive situation, a language game, that meaning and a language can emerge (\citet{Baroni2020}).
% Multiple deep neural models, called agents, communicate with a set of symbols to solve a task.
% Guided by the task and the rules, how agents can generate and interpret symbols, a language can emerge.

\subsubsection{Properties of the emerged language}
\label{sec:properties-el}


\subsubsection{Referring to objects and grounding}
\label{sec:referring}

\subsection{Dataset}
\cmtDK[inline]{\\
    - CLEVR \\
    ---- 3D objects \\
    ---- simple objects \\
    - CLEVR baselines \\
}

\subsection{Computational methods}

