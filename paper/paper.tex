% This must be in the first 5 lines to tell arXiv to use pdfLaTeX, which is strongly recommended.
%\pdfoutput=1
% In particular, the hyperref package requires pdfLaTeX in order to break URLs across lines.

\documentclass[11pt]{article}

% Remove the "review" option to generate the final version.
\usepackage[review]{acl}

% Standard package includes
\usepackage{times}
\usepackage{latexsym}

% For proper rendering and hyphenation of words containing Latin characters (including in bib files)
\usepackage[T1]{fontenc}
% For Vietnamese characters
% \usepackage[T5]{fontenc}
% See https://www.latex-project.org/help/documentation/encguide.pdf for other character sets

% This assumes your files are encoded as UTF8
\usepackage[utf8]{inputenc}

% This is not strictly necessary, and may be commented out,
% but it will improve the layout of the manuscript,
% and will typically save some space.
\usepackage{microtype}

% If the title and author information does not fit in the area allocated, uncomment the following
%
%\setlength\titlebox{<dim>}
%
% and set <dim> to something 5cm or larger.

\title{Instructions for *ACL Proceedings}

% Author information can be set in various styles:
% For several authors from the same institution:
% \author{Author 1 \and ... \and Author n \\
%         Address line \\ ... \\ Address line}
% if the names do not fit well on one line use
%         Author 1 \\ {\bf Author 2} \\ ... \\ {\bf Author n} \\
% For authors from different institutions:
% \author{Author 1 \\ Address line \\  ... \\ Address line
%         \And  ... \And
%         Author n \\ Address line \\ ... \\ Address line}
% To start a seperate ``row'' of authors use \AND, as in
% \author{Author 1 \\ Address line \\  ... \\ Address line
%         \AND
%         Author 2 \\ Address line \\ ... \\ Address line \And
%         Author 3 \\ Address line \\ ... \\ Address line}

\author{Dominik Künkele \\
  Affiliation / Address line 1 \\
  Affiliation / Address line 2 \\
  Affiliation / Address line 3 \\
  \texttt{dominik.kuenkele@outlook.com} \\\And
  Second Author \\
  Affiliation / Address line 1 \\
  Affiliation / Address line 2 \\
  Affiliation / Address line 3 \\
  \texttt{email@domain} \\}

\begin{document}
\maketitle
\begin{abstract}

\end{abstract}

\section{Introduction}
TODO:
\begin{itemize}
  \item how
\end{itemize}
\section{Background and previous work}

\section{Materials and methods}
TODO:
\begin{itemize}
  \item creation of dataset (CLEVR)
        \begin{itemize}
          \item multiple 'real' objects in scene
          \item 3 attributes (color, size, shape) differentiate objects
          \item using 'dale' setup to uniquely identify target object
        \end{itemize}
  \item building a language game using EGG
  \item based on feature extractors ResNet/VGG
  \item
  \item setup of discriminating game of objects in image
  \item message encoder/decoder is auto-encoder
  \item
\end{itemize}

\section{Results}
TODO:
\begin{itemize}
  \item DaleTwo:
        \begin{itemize}
          \item small hidden/embedding dims, small vocab -> high accuracy
          \item high hidden/embedding dims, small vocav -> low accuracy
        \end{itemize}
  \item DaleFive:
        \begin{itemize}
          \item small hidden/embedding dims, small vocab -> low accuracy
          \item small hidden/embedding dims, bigger vocab -> higher accuracy
        \end{itemize}
  \item ... test different hidden/embedding dims
  \item ... test 3/4 objects
\end{itemize}

\section{Discussion}
TODO:
\begin{itemize}
  \item maybe calculation of loss (multiplicating instead of summing loss per token), unlikely, since sequence length short -> shouldn't result in big differences
  \item reducing dims of image better the increasing dims of message, increasing dims is not learnable for models
  \item Vocab:
        \begin{itemize}
          \item vocabulary could describe attributes of target image (non-discriminative) or describe only differences (discriminative)
          \item in second case, two images is a far easier task than five images. Hence, much lower accuracy
        \end{itemize}
\end{itemize}

\section{Conclusions and further work}

\section*{Acknowledgements}

% Entries for the entire Anthology, followed by custom entries
%\bibliography{anthology,custom}

\appendix

\section{Example Appendix}


\end{document}